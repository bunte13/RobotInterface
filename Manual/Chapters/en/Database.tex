%%%
%
% $Autor: Wings $
% $Datum: 2021-05-14 $
% $Pfad: GitLab/CornerBLending $
% $Dateiname: Hints
% $Version: 4620 $
%
% !TeX spellcheck = de_DE/GB
%
%%%



\chapter{Database Population}
Populating the database is essential for your Robot Interface to work,
by populating the database with the default format provided in the ***,
you enable your functions to connect with the commands needed to execute the function, you also provide information about the libraries and the Category this function belongs to.\\
\subsection{How to Populate the Database:}
\begin{enumerate}
	\item In the navigation bar find the "Upload" section and after clicking it a window will open where you can enter an excel file.
	\item Upload the Excel file that you can find ***.
	\item After Uploading the Excel file all the functions and commands will be loaded in the database.
	\item This can be confirmed by going to the Functionalities section in the navigation bar and clicking on it.
	\item Here you can see all the functionalities that the database has been populated with, alongside the libraries and the Category it belongs to.
\end{enumerate}
In the details section for every functionality you can find:
\begin{itemize}
	\item The category this function belongs to.
	\item The libraries included in this function.
	\item All the commands that are used to execute this function.
\end{itemize}
